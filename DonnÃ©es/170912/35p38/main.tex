\documentclass{article}
\usepackage[utf8]{inputenc}
\usepackage[T1]{fontenc}
\usepackage[frenchb]{babel}
\usepackage{tkz-tab}

\title{Un exemple de tableau de signe}
\author{Thomas Gire }

\begin{document}

\maketitle

\section*{35 page 38}

Soit $f(x)=-2(x-1)(x-3)$

\[
\begin{tikzpicture}
\tkzTabInit[lgt=2,espcl=1.5]
{$x$                                   /1,
$-2$                            /1,
$x-1$                   /1,
$x-3$ /2,
$f(x)$ /2}
{$-\infty$  , $1$  , $3$ ,$+\infty$}
\tkzTabLine{ ,-,t,-,t,-,}
\tkzTabLine{ ,-,z,+,t,+,}
\tkzTabLine{ ,-,t,-,z,+,}
\tkzTabLine{ ,-,z,+,z,-,}
\end{tikzpicture}
\]

L'équation $-2(x-1)(x-3)>0$ a donc pour ensemble de solutions solution $S=]1;3[$.
\end{document}


